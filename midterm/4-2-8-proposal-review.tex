\subsubsection{健康ウォーキングマップのアプリ化の提案の振り返り}
健康ウォーキングマップの現状や目標、コースの作成体制といった函館市のホームページからではわからない詳細な情報が聞けた。元々健康ウォーキングマップは函館市民の健康増進を目的に作らているコンテンツであり、観光目的では無いというご指摘を頂いた。またログ機能においては、何度も同じコースを歩く人でもモチベーションの向上が期待できる機能が欲しいという要望があった。アプリ化した後の運用・保守については不安を抱いており、公開されるものであれば継続して使えるものにして欲しいという要望もあった。そのため、データ利用許可については、こちらの継続体制が決まってから協議するという事になった。提案の後、本チームはプロジェクト学習としての期間が終わっても大学院進学を希望するメンバーが在学中の3年間は運用・保守を行うことをメールで伝えた。後日、無事データ利用許可を頂く事ができた。この提案によってログ機能の再考案やマップを観光面で推すといった新しい課題ができた。アプリをいきなり実装せずに提案しご意見を頂けた事は、大きな手戻りを起こるのを防ぎアプリの仕様改善に繋がった。

\bunseki{仲松聡}
