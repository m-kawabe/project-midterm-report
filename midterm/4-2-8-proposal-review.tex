\subsubsection{健康ウォーキングマップのアプリ化の提案の振り返り}
健康ウォーキングマップの現状や目標、コースの作成体制といった函館市の
ホームページからではわからない詳細な情報が聞けた。元々健康ウォーキン
グマップは函館市民の健康増進を目的に作らているコンテンツであり、観光
目的では無いという指摘を得た。またログ機能においては、何度も同じコー
スを歩く人でもモチベーションの向上が期待できる機能が欲しいという要望
があった。アプリ化した後の運用・保守については不安を抱いており、公開
されるものであれば継続して使えるものにして欲しいという要望もあった。
そのため、データを用いる許可については、こちらの継続体制が決まってか
ら協議するという事になった。提案の後、本グループはプロジェクト学習と
しての活動期間が終わっても大学院進学を希望するメンバーが在学中の3年間
は運用・保守を行うことをメールで伝えた。後日、無事データ利用許可を頂
く事ができた。この提案によってログ機能の再考案や健康ウォーキングマッ
プが健康増進を目的としている事を考慮した公開をするといった新しい課題
ができた。アプリをいきなり実装せずに提案しご意見を頂けた事は、大きな
手戻りを起こるのを防ぎアプリの仕様改善に繋がった。

\bunseki{仲松聡}
