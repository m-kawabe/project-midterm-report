\begin{jabstract}
%本プロジェクトでは観光・業務・教育の三つのテーマに分かれて、それぞれの現状に対しての特効薬になり得るタブレットアプリ開発を行う。

地方都市では観光産業が低迷傾向にあり、
函館市も例外ではなく観光客の年平均人数が緩やかな減少傾向にある。
一方で従来の物見遊山的な観光旅行に対して、
これまで観光資源としては気付かれていなかったような地域固有の資源を新たに活用し、
体験型・交流型の要素を取り入れた旅行形態が注視されている。
本グループは函館市の観光に関する取り組みについて調査を行ったところ、
健康づくりのためのコンテンツ「健康ウォーキングマップ」を発見した。
この健康ウォーキングマップは、函館市民の健康増進を目的に作られた
手書きのウォーキングマップである。
現状では健康ウォーキングマップは地域固有の魅力的な情報を持ち、
その魅力をより上手く活用することが期待できる。
健康ウォーキングマップをアプリ化することでその魅力を引き出し、
見やすくすることが本グループの課題である。
この課題を解決し、
健康ウォーキングマップにあるヘルスツーリズムの可能性をアプリで
引き出すことが本グループの目的である。
課題解決へ向けて、本グループはウォーターフォール型の開発工程で、
健康ウォーキングマップの良さを引き出したウォーキングアプリ、
「はこウォーク」を開発した。
はこウォークは健康ウォーキングマップの良さを引き出すための機能を実装しており、
ユーザに使用してもらうことで、より健康ウォーキングマップの
魅力が伝わるようになっている。
また、このアプリは2015年1月中に公開予定である。

\begin{jkeyword}
健康ウォーキングマップ、観光、健康
\end{jkeyword}
\bunseki{仲松聡}
\end{jabstract}
